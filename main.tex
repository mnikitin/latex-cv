%% start of file `template.tex'.
%% Copyright 2006-2013 Xavier Danaux (xdanaux@gmail.com).
%
% This work may be distributed and/or modified under the
% conditions of the LaTeX Project Public License version 1.3c,
% available at http://www.latex-project.org/lppl/.


\documentclass[11pt,a4paper,roman]{moderncv}        % possible options include font size ('10pt', '11pt' and '12pt'), paper size ('a4paper', 'letterpaper', 'a5paper', 'legalpaper', 'executivepaper' and 'landscape') and font family ('sans' and 'roman')

% modern themes
\moderncvstyle{banking}                            % style options are 'casual' (default), 'classic', 'oldstyle' and 'banking'
\moderncvcolor{blue}                                % color options 'blue' (default), 'orange', 'green', 'red', 'purple', 'grey' and 'black'
%\renewcommand{\familydefault}{\sfdefault}         % to set the default font; use '\sfdefault' for the default sans serif font, '\rmdefault' for the default roman one, or any tex font name
\nopagenumbers{}                                  % uncomment to suppress automatic page numbering for CVs longer than one page

% character encoding
\usepackage[utf8]{inputenc}
\usepackage{fontawesome}
\usepackage{fontspec}
\usepackage{tabularx}
\usepackage{ragged2e}
% if you are not using xelatex ou lualatex, replace by the encoding you are using
%\usepackage{CJKutf8}                              % if you need to use CJK to typeset your resume in Chinese, Japanese or Korean

% adjust the page margins
\usepackage[scale=0.8]{geometry}
\usepackage{multicol}
%\setlength{\hintscolumnwidth}{3cm}                % if you want to change the width of the column with the dates
%\setlength{\makecvtitlenamewidth}{10cm}           % for the 'classic' style, if you want to force the width allocated to your name and avoid line breaks. be careful though, the length is normally calculated to avoid any overlap with your personal info; use this at your own typographical risks...

\usepackage{import}

% personal data
\name{Mikhail}{Nikitin}
% \title{Curriculum Vitae}                               % optional, remove / comment the line if not wanted
\address{}{}{}% optional, remove / comment the line if not wanted; the "postcode city" and and "country" arguments can be omitted or provided empty
% \phone[mobile]{909-839-3097}                   % optional, remove / comment the line if not wanted
% \phone[fixed]{01234 123456}                    % optional, remove / comment the line if not wanted
%\phone[fax]{+3~(456)~789~012}                      % optional, remove / comment the line if not wanted
% \email{xpan1@swarthmore.edu}                               % optional, remove / comment the line if not wanted
% \homepage{shawnpan.me}                         % optional, remove / comment the line if not wanted
% \extrainfo{}                 % optional, remove / comment the line if not wanted
%\photo[64pt][0.4pt]{picture}                       % optional, remove / comment the line if not wanted; '64pt' is the height the picture must be resized to, 0.4pt is the thickness of the frame around it (put it to 0pt for no frame) and 'picture' is the name of the picture file
%\quote{Some quote}                                 % optional, remove / comment the line if not wanted

% to show numerical labels in the bibliography (default is to show no labels); only useful if you make citations in your resume
%\makeatletter
%\renewcommand*{\bibliographyitemlabel}{\@biblabel{\arabic{enumiv}}}
%\makeatother
%\renewcommand*{\bibliographyitemlabel}{[\arabic{enumiv}]}% CONSIDER REPLACING THE ABOVE BY THIS

% bibliography with mutiple entries
%\usepackage{multibib}
%\newcites{book,misc}{{Books},{Others}}
  
\newcommand*{\customcventry}[7][.25em]{
  \begin{tabular}{@{}l} 
    {\bfseries #4}
  \end{tabular}
  \hfill% move it to the right
  \begin{tabular}{l@{}}
     {\bfseries #5}
  \end{tabular} \\
  \begin{tabular}{@{}l} 
    {\itshape #3}
  \end{tabular}
  \hfill% move it to the right
  \begin{tabular}{l@{}}
     {\itshape #2}
  \end{tabular}
  \ifx&#7&%
  \else{\\%
    \begin{minipage}{\maincolumnwidth}%
      \small#7%
    \end{minipage}}\fi%
  \par\addvspace{#1}}

\newcommand*{\customcvproject}[4][.25em]{
%   \vfill\noindent
  \begin{tabular}{@{}l} 
    {\bfseries #2}
  \end{tabular}
  \hfill% move it to the right
  \begin{tabular}{l@{}}
     {\itshape #3}
  \end{tabular}
  \ifx&#4&%
  \else{\\%
    \begin{minipage}{\maincolumnwidth}%
      \small#4%
    \end{minipage}}\fi%
  \par\addvspace{#1}}

\setlength{\tabcolsep}{12pt}

%----------------------------------------------------------------------------------
%            content
%----------------------------------------------------------------------------------
\begin{document}
%\begin{CJK*}{UTF8}{gbsn}                          % to typeset your resume in Chinese using CJK
%-----       resume       ---------------------------------------------------------
\makecvtitle
\vspace*{-23mm}

\begin{center}
\begin{tabular}{ c c c }
 \faEnvelopeO\enspace mikhail.nikitin92@gmail.com & \faGithub\enspace \href{https://github.com/mnikitin}{mnikitin} & \faMobile\enspace +79653804518\\  
\end{tabular}
\end{center}

\section{EDUCATION}
{\customcventry{2009 - 2014}{Computational Mathematics and Cybernetics Department}{Specialist Degree}{Moscow State University}{}{Graphics \& Media Lab, Computer Vision Group}}
{\customcventry{Expected Graduation: 2022}{Computational Mathematics and Cybernetics Department}{PhD Degree}{Moscow State University}{}{Graphics \& Media Lab, Computer Vision Group}}

\section{EXPERIENCE}

{\customcventry{July 2012 - Sept 2013}{Junior Computer Vision Researcher}{Video Analysis Technologies (Tevian)}{Moscow, part-time}{}
{\begin{itemize}
  \item Face Recognition and Face Features Detection based on hand-crafted features
\end{itemize}
}

{\customcventry{Oct 2013 – Dec 2014}{Software Developer and Researcher}{Video Analysis Technologies (Tevian)}{Moscow}{}
{\begin{itemize}
  \item Face Recognition, Face Features Detection and Face Quality Assessment based on hand-crafted features
  \item Data annotation management
  \item Implemented some investigated algorithms in C++ for integration into the Tevian FaceSDK
  \item Developed multithreaded facial analysis system for one of the customers
\end{itemize}
}

{\customcventry{Jan 2015 – Dec 2018}{Computer Vision Researcher}{Video Analysis Technologies (Tevian)}{Moscow}{}
{\begin{itemize}
  \item Developed new or improved existed algorithms in Tevian FaceSDK:
  \begin{itemize}
  	\item Face Recognition (continuous improvement)
  	\item Face Quality Assessment (simple face rotation classifiers and brand new training method for overall FQA)
  	\item Anti-spoofing (using synthetic data created one of the first deep learning based algorithms in the world)
  	\item Face detector false positives filter
  \end{itemize}
  \item Developed several vehicle classifiers for analysis of data from traffic monitoring cameras
  \item Created some auxiliary algorithms for automatic store's shelves fullness analysis
  \item As scientific supervisor worked on the following projects:
  \begin{itemize}
  	\item Person re-identification
  	\item TV broadcast analysis
  \end{itemize}
\end{itemize}
}

{\customcventry{Jan 2019 – Now}{Lead Computer Vision Researcher}{Video Analysis Technologies (Tevian)}{Moscow}{}
{\begin{itemize}
  \item Lead team of researches who work on further improvemenet of Tevian FaceSDK
  \item Team research projects:
  \begin{itemize}
  	\item Face anti-spoofing (RGB and depth based algorithms)
  	\item Facial attributes classification (eyewear, headwear, facial hair, hair type, hair color)
  	\item Face-based demography analysis (gender, age, ethnicity)
  	\item Face Quality Assessment (fr-based overall quality estimator)
  	\item Face Recognition:
  	\begin{itemize}
  		\item general improvement using new architecture blocks and training methods
  		\item domain-specific improvements (medical masks, demography biases, high FPR with low quality faces)
  		\item data refinement (test datasets and large-scale training dataset)
  		\item face images clustering
  		\item speeding up facial features extraction and matching 
  	\end{itemize}
  \end{itemize}
\end{itemize}
}

%\section{PROJECTS}
%
%{\customcvproject{Amazing Project 1}{Aug 2017 - Present}
%  {\begin{itemize}
%    \item Ran custom Minecraft mods on CentOS and Debian servers
%    \item Reverse engineered obfuscated Java using a decompiler and reflection
%  \end{itemize}
%  }
%}
%
%{\customcvproject{Amazing Project 2}{Feb 2016 – May 2017}
%{\begin{itemize}
%  \item Began a group to create the first 36 hour student-run hackathon ever held at Illinois
%  \item Led 50 staff to raise \$175,000 for HackIllinois, managing relationships with 60+ sponsors
%  \item Plan events to increase social interaction among ACM members, as well as events to foster the spirit of computer science at Illinois
%\end{itemize}
%}

%{\customcvproject{Amazing Project 3}{Jan 2015 – May 2015}
%{\begin{itemize}
%  \item Assist in recruiting potential and admitted students to the UIUC computer science program
%  \item Attend information sessions, admitted student Q\&A’s, and student lunches to generate interest in the Illinois Computer Science department
%\end{itemize}
%}
%}

\section{ADDITIONAL}
\begin{minipage}{\maincolumnwidth}%
	\small{
    	\begin{itemize}
    	  \item For several years was the only researcher in company who worked on the face recognition algorithm
    	  \item Tevian's FR engine is used in many real-life projects across the world, and its high quality additionally confirmed by the NIST FRVT (3rd among Russian vendors)
    	  \item In 2015 won olympiad on computer vision organized by VisionLabs. As a reward, did short-term internship at INRIA Paris where worked on video face recognition under supervision of Ivan Laptev
    	  \item As PhD student, taught programming courses at the university, and was an unofficial scientific supervisor for several Lab students
		\end{itemize}}%
\end{minipage}%
      
}
% Publications from a BibTeX file without multibib
%  for numerical labels: \renewcommand{\bibliographyitemlabel}{\@biblabel{\arabic{enumiv}}}% CONSIDER MERGING WITH PREAMBLE PART
%  to redefine the heading string ("Publications"): \renewcommand{\refname}{Articles}
\renewcommand{\refname}{PUBLICATIONS}
\nocite{*}
\bibliographystyle{plain}
\bibliography{publications}                        % 'publications' is the name of a BibTeX file

\begin{itemize}
	\item Nikitin M. et al. \textit{Face quality assessment for face verification in video.} In GraphiCon'2014
	\item Nikitin M. et al. \textit{Neural network model for video-based face recognition with frames quality assessment}. In Computer Optics 41 (5), 2017
	\item Nikitin M. et al. \textit{Face anti-spoofing with joint spoofing medium detection and eye blinking analysis}. In Computer Optics 43 (4), 2019
	\item Nikitin M. et al. \textit{Pairwise Ranking Distillation for Deep Face Recognition}. In GraphiCon'2020
\end{itemize}

% Publications from a BibTeX file using the multibib package
%\section{Publications}
%\nocitebook{book1,book2}
%\bibliographystylebook{plain}
%\bibliographybook{publications}                   % 'publications' is the name of a BibTeX file
%\nocitemisc{misc1,misc2,misc3}
%\bibliographystylemisc{plain}
%\bibliographymisc{publications}                   % 'publications' is the name of a BibTeX file

%-----       letter       ---------------------------------------------------------

\end{document}


%% end of file `template.tex'.

